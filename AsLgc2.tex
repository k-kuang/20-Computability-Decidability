\documentclass[UTF8, 9pt, a4paper]{ctexart}
\usepackage[utf8]{inputenc}
\usepackage{multicol}
\usepackage{longtable}
\usepackage{multicap}
\usepackage{tabu}
\usepackage{amsmath}
\usepackage[left = 15mm, right = 15mm, top = 25mm, bottom= 20mm]{geometry}
\usepackage{stmaryrd, amssymb}
\usepackage{comment}
\usepackage{changepage}

\usepackage{pgf}
\usepackage{tikz}
\usetikzlibrary{positioning} % Necessary to use right=5cm of A
\usetikzlibrary{arrows,automata}
\newcommand{\ksec}[2]{\noindent \textbf{\large #1} #2\par}
\newcommand{\ksolve}{\noindent\textbf{\large Solution: }\par}
\newcommand{\kspace}{\vspace{0.5cm}}
\usepackage{fancybox}

\begin{document}
	
	\ksec{1. }{存在一个闭集A使得$ (A)^{(\omega + 1)} = \emptyset \neq (A)^{(\omega)} $ }
	按如下步骤定义集合:
	\begin{itemize}
		\item $ A_0 \triangleq \{0\} $
		\item $ A_1 \triangleq \{1,\dfrac{1}{2}, \dfrac{2}{3}, ..., \dfrac{n}{n+1}, ... \} $ (是闭集)
		\item $ A_2 \triangleq \{2\} \cup (1 + \dfrac{1}{2} \times A_1) \cup (1+\dfrac{1}{2}+ \dfrac{1}{6} \times A_1 )\cup... \cup (1 + \dfrac{n}{n+1} + \dfrac{1}{(n+1)(n+2)} \times A_1) \cup... $ (是闭集)\\
		......
		\item $ A_k \triangleq \{n\} \cup \bigcup\limits_{i \in \omega} (k + \dfrac{i}{i+1} + \dfrac{1}{(i+1)(i+2)} \times A_{k-1}) $\\
		......
	\end{itemize}
	定义$ A_\omega \triangleq \bigcup\limits_{\omega}^{} A_i $(是闭集),$ A_\omega $中有$ \omega^\omega $级的元素。$ A_\omega^{(\omega)} = \emptyset $. 故还需进行一次压缩, 定义$ A $:
		$$ A = \{ \dfrac{t}{t+1} \mid t \in A \} \cup \{1\} \qquad (\texttt{是闭集})$$  
		
	有$ A^{(w)} = \{1\} $, $ A^{(\omega + 1)} = \emptyset $, 故存在闭集A使得题设成立。
	
	
	
	\vspace{0.5cm}
	
	\ksec{2. }{若把$ \overset{=}{A} \leq \overset{=}{B} $定义为存在B到A的满射,则康托伯恩斯坦定理是否依然成立?}
	
	成立.\par
	设$ f: A \rightarrow B, g: B \rightarrow A $均为满射。
	\begin{itemize}
		\item 因$ \forall b \in B:\ \exists a\in A:\ f(a) = b $, 不妨构造:$ S_b = \{ a \mid a\in A \land f(a) = b\} $, 易知$ A = \bigcup\limits_{b \in B} S_b  $且$ \forall b_1 \neq b_2:\ S_{b_1} \cap S_{b_2} = \emptyset $. 根据选择公理,存在一个集合$ A' $,对每个$ S_b $,$ A' $包含且仅包含一个$ S_b $中的元素。构造$ f': B \rightarrow A' $, 使得$ \forall b \in B:\ \exists a \in A':\ f'(b) = a$,即$ f' $是一个单射。
		
		\item 同理构造$ g' $
	\end{itemize}

	根据康托伯恩斯坦定理,存在$ A $到$ B $的单射$ g' $和$ B $到$ A $的单射$ f' $,因此存在$ A $到$ B $的双射,证毕。
	
	\begin{comment}
	$ \overset{=}{A} \leq \overset{=}{B} $: if 存在A到B的单射。
	有穷的情况下必然正确,若两集合数量不一样,总有一边无法达成满射。
	无穷:f映满B中所有元素,去掉重复映射,仅保留一个, 得到$ f':A' \rightarrow B $。g映满A中所有元素s,同理处理,得到$ g':B' \rightarrow A $. 显然:$ |B| = |B'| = |A| = |A'| $, 且$ g'\ f' $都是双射。
	单且满为双。
	\end{comment}	
	
	\vspace{0.5cm}
	
	\ksec{3. }{若$ \alpha, \beta $是序数,且$ f: \alpha \rightarrow \beta $是保序双射(同构), 则$ \alpha = \beta $且$ f $是恒等映射}
	
	\ksolve
	不妨定义:$ g = f^{-1} $,易知$ g $也是保序双射
	\begin{itemize}
		\item 证明子命题1:$ \forall a \in \alpha:\ a \subseteq f(a) $.\par
		
		定义$ T = \{ t \mid f(t) \in t \} $, 反设$ T \neq \emptyset $. 由良序公理,T中有极小元$ t_0 $. \par
		
		由$t_0 \in t$,故$ f(t_0) \in t_0 $。设$ t_0' = f(t_0) $, 由$ f $保序且一一且$ t_0' \in t_0 $, 有$ f(t_0') \subset f(t_0) $, 由序数性质知$ f(t_0') \in f(t_0) $,即$ f(t_0') \in t_0' $, 故$ t_0' \in T $. 与$ t_0 $是极小元矛盾。故子命题成立。\par
		同理可证子命题2:$ \forall b \in \beta:\ b \subseteq g(b)  $
		
		
		\item 证$ \alpha = \beta $. 反设$ \alpha \neq \beta $, 则$ \alpha \in \beta $或$ \beta \in \alpha $, 不妨假设$ \beta \in \alpha $.\par
		由$ \beta \in \alpha $, 知$ \exists \gamma:\ \gamma\in \alpha \land \gamma \notin \beta $. 由三歧性即$ \gamma = \beta \lor \gamma \in \beta $。又因$ f(\gamma) \in \beta $,得$ f(\gamma) \in \gamma $. 与子命题矛盾。\par
		同理可证$ \alpha \in \beta $的情况有矛盾。故$ \alpha = \beta $.
		
		\item 证$ f $恒等\par
		假设$ f $不恒等,即$ \exists a:\ f(a) \neq a $, 由三歧性有$ f(a) \in a $或$ a \in f(a) $
		\begin{itemize}
			\item $ f(a) \in a $: 与子命题1矛盾
			\item $ a \in f(a) $: 设$ f(a) = b $, 则$ g(b) \in b $与子命题2矛盾
		\end{itemize}
		故$ \forall a: f(a) = a $, f是恒等映射。
	\end{itemize}
	
	
	
	
	
	

	(保序:$ f: S \rightarrow T $, $ \forall s_1\ s_2 \in S:\ s_1 \subseteq  s_2 \implies f(s_1) \subseteq f(s_2)\ $
	
	\kspace
	
	\begin{comment}
		
	
	暂时将“序”使用$ \leq, \geq $符号\\
	$ \forall x \ y\in \alpha:\ x \leq y \implies f(x) \leq f(y) $\\
	$ \forall x \ y\in \beta:\ x \leq y \implies f^{-1}(x) \leq f^{-1}(y) $\\
	
	考虑1,2 -> 3,4, 利用序数的性质证明$ \forall x\ y\in \alpha:\ f(x) = x $, 反证?能推出什么矛盾?
	
	$ \forall x \in \alpha:\ x \in \beta $\\
	$ \forall x \in \beta:\ x \in \alpha $\\
	$ \alpha \subseteq \beta \land \beta \subseteq \alpha $\\
	$ \forall a \in \texttt{dom}(f):\ f(a) = a $,由恒等映射的性质,其是双射。\\
	
	\end{comment}
	
	
	\vspace{0.5cm}
	
	\ksec{4. }{任何一个实数集合都是一个可数集与一个无孤立点集的不交并}
%	$ \overset{=}{A}  \leq \aleph_1 $,
	
	设$ A \subseteq \mathbb{R} $为实数集合,记$ A^{(k)} $为$ A $第$ k $次求导的结果。
	
	\begin{itemize}
		\item $ \exists \alpha: \forall \alpha > \gamma: A^{(\alpha)} = (A^{(\alpha)})' = A^{(\gamma)}$\\
		证:
		\begin{itemize}
			\item $\gamma$存在:每求一次导必然去掉至少一个有理数对$ (p, q) $, 且对任意$ x \neq y $且$ x\ y $为孤立点,有理数对不相交(否则与$ x \neq y $矛盾),故$ (p_x, q_x) \neq (p_y,q_y) $. 由此$ \alpha < \aleph_1 $(否则与$ \overset{=}{\mathbb{Q}} = \aleph_0 $矛盾)。
			\item 证明对所有$ \alpha' > \alpha $成立:\\
			若$ \alpha $是后继序数,则存在$ \beta \geq \alpha$, $ A^{(\alpha)} = A^{(\beta + 1)} $, 由归纳假设知$ A^{(\alpha)} = (A^{(\beta)})' = (A^{(\gamma)})' = A^{(\gamma)} $\\
			若$ \alpha' $是极限序数, 则$ A^{(\alpha)} = \bigcup_{\beta < \gamma} A^{(\beta)} = \bigcup_{\gamma \leq \beta < \alpha} A^{(\beta)}$。由归纳假设知满足条件的$ A^{(\beta)} = A^{(\gamma)} $,故$ A^{(\alpha)} = A^{(\gamma)} $
		\end{itemize}
		$ A^{(\gamma)} $是无孤立点集(否则与$ A^{(\gamma)} = A^{(\gamma + 1)} $矛盾)
		
		
		\item $ B_\alpha = \{ x \mid x \in A^{\alpha} \land x \notin A^{(\alpha + 1)} \} $, $ \bigcup\limits_{\alpha < \gamma} B_\gamma  $是可数集。\\
		证:定义$ C_\alpha = \{ (p, q) \mid x\texttt{是}A^\alpha\texttt{中的孤立点且}(p,q) \cap A^\alpha = \{ x\} \} $. 可建立$ \bigcup\limits_{\alpha < \gamma} B_\alpha  $到$\bigcup\limits_{\alpha < \gamma} C_\alpha  $上的双射, 由因为$ \bigcup\limits_{\alpha < \gamma} C_\alpha $的势小于等于$ \mathbb{Q} \times \mathbb{Q} $的势,故$ \bigcup\limits_{\alpha < \gamma} B_\alpha  $可数。
		
		\item 而$ A = (\bigcup\limits_{\alpha < \gamma} B_\alpha) \uplus A^{(\gamma)}$, 其中前者是可数集,后者是无孤立点集。证毕
	\end{itemize}
	\begin{comment}
	记$ T = \{x \mid x \in A \land x\texttt{是}A\texttt{中的孤立点}\} $
	
	
	
	分情况讨论证明如下:
	不是闭集,也不是开集的集合。$ [5, 6) \cup \{3\} $不是闭集,有孤立点。
	\begin{itemize}
		\item 对不可数且是闭集的实数集合S,由完备集定理知其是一个可数集与一个完备集的不交并,完备集属于无孤立点集合,故命题成立
		\item 对不可数且非闭集的实数集合S,开集本身没有孤立点,$ S = \emptyset \cup S $, S无孤立点集,$ \emptyset $可数,故命题成立
		\item 对可数实数集合$ S $:$ S = S \cup \emptyset $, S可数,$\emptyset$无孤立点,故命题成立
	\end{itemize}
	综上所述,命题成立
	\end{comment}
	
	\vspace{0.5cm}
	
	\ksec{5. }{证明良序公理等价于对每个集合$ X $,存在一个函数$ f: \mathcal{P}(X) \rightarrow X $, 使得对所有的$ Y \subseteq X $, $ Y \neq \emptyset \implies f(Y) \in Y$}
	\ksolve
	
	良序公理:
		$$ \forall S:\ \forall T \subseteq  S:\exists t \in T :\ \forall t'\in T:\ t \subseteq t' $$
	
	选择函数公理:
		$$ \forall S:\ \exists f: \mathcal{P}(S) \rightarrow S:\ \forall T \subseteq S:\ T \neq \emptyset \implies P(T) \in T $$
	
	\begin{itemize}
		\item 良序公理 $\rightarrow$ 选择函数公理:
		$ \forall Y \in \mathcal{P}(X)$且$\ Y \neq \emptyset $,由良序公理可从中选出其极小元y。定义$ f: \mathcal{P}(X) \rightarrow X $,$ \forall Y \in \texttt{dom}(f) \land Y \neq \emptyset$, $ f(Y) \texttt{等于Y中的极小元}$. $ f $是符合条件的选择函数。
		
		\item 选择函数公理 $ \rightarrow $ 良序公理:
			只需找到某个序数$ \alpha $, 建立$ \alpha $到$ X $的一一映射即可。即将$ \alpha_0, \alpha_1, ..., \alpha_n, .... \in \alpha $一一映到$ X $上, 定义G如下:
			
			$$ \begin{array}{ll}
				G(\alpha_0) = f(X) \\
				G(\alpha_1) = f(X \setminus \{ G(\alpha_0)\}) \\
				G(\alpha_2) = f(X \setminus \{ G(\alpha_0), G(\alpha_1) \}) \\
				......\\
				G(\alpha_k) = f(X \setminus \{ G(\alpha_i) \mid \alpha_i \in \alpha_k \}) \\
				......
			\end{array}$$
		由$ \forall Y:\ f(Y) \in Y $, $ G $最终会将序数列与X建立一一映射,因此知X是良序的。
		
		
		\begin{comment}
		$$\begin{array}{l}
			\texttt{对}X: \exists (f_0 \in \mathcal{P}(X) \rightarrow X) :\ \forall Y \subseteq X:\ Y \neq \emptyset \implies f_0(Y) \in Y\\
			\texttt{对}\mathcal{P}(X): \exists (f_1 \in \mathcal{P}(\mathcal{P}(X)) \rightarrow \mathcal{P}(X)) :\ \forall Y \subseteq \mathcal{P}(X):\ Y \neq \emptyset \implies f_1(Y) \in Y\\
			\texttt{对}\mathcal{P}(X): ......\\
			\texttt{对}\mathcal{P_\omega}(X): \exists (f_\omega \in \mathcal{P_{\omega}^1} (X) \rightarrow \mathcal{P}_\omega(X)) :\ \forall Y \subseteq \mathcal{P}_\omega(X):\ Y \neq \emptyset \implies f_\omega(Y) \in Y\\
			......
		\end{array}
		$$
		定义$ f: \mathcal{P}_\omega(X) \rightarrow X $
		$$ f = f_0 \circ f_1 \circ ... \circ f_\omega $$
		
		从\texttt{dom}($ f $)中取一个Y使得$ \forall y\in Y:\ y \texttt{ is ordinal} \land \forall y' \notin Y:\ y' \texttt{ is not ordinal} $, 由命题知$ \forall y \in Y:\ f(Y) \in Y $
		$ f' $为限制定义域为$ Y $的f, 
		证明$ f' $是一个双射。
		\end{comment}
		
		
	\end{itemize}
	综上所述,二者等价。
	
	\vspace{0.5cm}

	\ksec{6. }{每个不可数的$ G_\delta $实数集都有一个非空的完备子集}
	某不可数$ G_\delta $集$ A $描述如下
		$$ A = \bigcap\limits_{n \in \omega} U_n \qquad \texttt{其中}U_i\texttt{为开集}$$
		
	\indent \ksec{a. 构造实数集[0,1]到$ A $的子集$ D $的映射$ h $}{}
	由第四题结论,$ A $可以表达为:
		$$ A = B \cup C \qquad \texttt{其中}B\texttt{是可数集,}C\texttt{为无孤立点集}$$
		
	$ C $不空(因$ A $不可数且$ B $可数)且$ C \subseteq \bigcap\limits_{n \in \omega} U_n $
	
	故其中至少有一点$ x_0 $,因此可以找到某个区间$L \subseteq U_0$(不妨令$ |L| < 1 $)使得
		$$ x_0 \in C \cap L $$

	由$ C $无孤立点,故必存在$ x_1 \neq x_0$使得$$ x_1 \in C \cap L $$
	
	知闭区间$ I = [x_0, x_1] \subseteq L $。将$ x_0, x_1 $分开,分别用区间$ L_0, L_1 $(不妨令$ |L_0|, |L_1| < \dfrac{1}{2} $)包含之即
	 $$ x_0 \in L_0 \land x_1 \in L_1 \land L_0 \cap L_1 = \emptyset $$
	
	同理可在$ L_0 $中找到除$ x_0 $外的一点$ x_{01} $, $ L_1 $中除$ x_1 $外一点$ x_{11} $, (为方便记名,上一层的点$ x_0 $更名为$ x_{00} $, $ x_1 $更名为$ x_{10} $, 每下一层,末尾加一个0) 继续分割下去......下图描述了这一过程,同一层集族中的集合互不相交:\\
	
	\begin{tikzpicture}[level/.style={sibling distance=80mm/#1}]
	\node [] (z){$I = [x_0 , x_1 ]$}
	child {node [] (a) {$I_0 = [x_{00}, x_{01}] $}
		child {node [] (b) {$ I_{00} = [x_{000} , x_{001}] $}
			child {node {$\dots$}}
			child {node {$\dots$}}
		}
		child {node [] (g) {$ I_{01} = [x_{010} , x_{011} ] $}
			child {node {$\dots$}}
			child {node {$\dots$}}
		}
	}
	child {node [] (j) {$I_1 = [x_{10}, x_{11}]$}
		child {node [] (k) {$ I_{10} = [x_{100} , x_{101} ] $}
			child {node {$\dots$}}
			child {node {$\dots$}}
		}
		child {node [] (l) {$ I_{11} = [ x_{110} , x_{111} ]$}
			child {node {$\dots$}}
			child {node {$\dots$}{
				child [grow=right] {node (q) {$ ... $} edge from parent[draw=none]
						child [grow=up] {node {$\dfrac{1}{4}$} edge from parent[draw=none]
							child [grow=up] {node (s) {$\dfrac{1}{2}$} edge from parent[draw=none]
								child [grow=up] {node (t) {$1$} edge from parent[draw=none]}}
						}
					}
				}
			}
		}};
	\end{tikzpicture}
	
	定义映射$ f:Z \rightarrow C $, 其中
		$$ f(z) = x_z $$
	\begin{itemize}
		\item 定义$ Z $:长度是可数无穷多的01字符串构成的集合($ Z = \{0,1\}^\omega $)
		\item 定义$ x_z $: 由$ \{ I_{z \upharpoonright n}\}$ \footnote{ $ I_{z\upharpoonright n} $: 设字符串$ z $截取前n位为$ z' $, 则$ I_{z \upharpoonright n} = I_{z'} $. 例如$ I_{ 0011 \upharpoonright 3} = I_{001} $} $ \ (n \in \omega) $是一个区间套,根据区间套定理,在实数系中存在唯一的点$ x_z $使得$ x_z \in I_{z \upharpoonright n} $
	\end{itemize}
	
%	$ \forall z \in Z:\ I_z \in \bigcap\limits_{n \in \omega} U_n$
%		 \qquad (x_z \in \bigcap\limits_{n \in \omega} I_{z \upharpoonright n}) 
	易知$ f $为单射(若$ z_0 \neq z_1 $则$ f(z_0) \neq f(z_1)$), 由$ f $定义知其为是满射。
	记$ D = \texttt{range}(f) $。
	
	定义$ g :[0,1] \rightarrow Z $, $ g(x) $为实数$ x $的二进制表示,易知$ g $是$ [0,1] $到$ Z $的同构。
	
	定义$ h = f \circ g $。

	\vspace{1cm}

	\indent \ksec{b. 证明$ h $是连续双射}{}
	因$ f, g $为双射,故$ h: [0,1] \rightarrow D $为双射
	
	任取$ y_1, y_2 \in [0,1] $, 对任意$ \xi > 0 $, 取$ \delta = 2^{ \lfloor \texttt{log}_2\xi \rfloor} $有:
	$$ |y_1 - y_2| \textless \delta \implies |h(y_1) - h(y_2)| \textless \xi $$
	
	故$ h $为$ [0,1] $上的连续函数。同理可证$ h^{-1} $为$ D $上的连续函数。

	\indent \ksec{c. 证明$ D $是完备集}{}
	
	\begin{itemize}
		\item $ D $无孤立点:任取$ D $中元素$ y $,在$ [0,1] $中取收敛于$ h^{-1}(y) $的数列$ \{x_n\} $, 因$ h $连续, 故$ \lim\limits_{n \rightarrow \infty} h(x_n) = y $, 故$ y $是$ D $中聚点。
		
		\item $ D $是闭集:对任意$ D $中的柯西列$ \{ y_n \} $, 由$ h $连续知$ \{h^{-1}(y_n)\} $也是柯西列,记$ \lim\limits_{n \rightarrow \infty}h^{-1}(y_n) = x_0 $, 因$ [0,1] $是闭集,故$ x_0 \in [0,1] $,因此$ h(x_0) $有定义且序列$\{y_n\} \rightarrow h(x_0) $, $ h(x_0) \in D $。因此$ D $内所有聚点都在$ D $中,故$ D $为闭集。
	\end{itemize}
	
	
	综上所述,$ D $不可数且完备,故任意不可数的$ G_\delta $实数集$ A $都有一非空完备子集。
	
	
	\begin{comment}
	
	只需证:每个不可数$ G_\delta $实数集都有一个uncountable close subset
		$$ G = \bigcap\limits_{i \in \omega }O_i $$
		
	只需证存在A,$ A \in G $且$ A \texttt{是闭集}$且$ A \texttt{不可数}$。
	
	$ G \texttt{不可数}$ 且 $ O_i \supset G $, 故$ O_i\texttt{不可数} $。
	\vspace{0.5cm}
	\end{comment}


\begin{comment}
	\ksec{附加练习: }{任给一个集合$ A $,其群有多少个}
	\ksec{附加练习: }{$ c([0, 1])=\{f \mid [0,1] \rightarrow \mathbb{P} \land f\texttt{连续}\} $,求$ |c| $}
	\ksec{arg1}{为何$ \kappa^\kappa\subseteq \mathcal{P}(\kappa \times \kappa) $}
	有理数是全序而不是良序。
	


	\begin{itemize}
		\item $ 2^\omega $
		\item borel sets
		\item perfect set proporty
		
	\end{itemize}	
\end{comment}
	

\end{document}