\documentclass[UTF8, 9pt, a4paper]{ctexart}
\usepackage[utf8]{inputenc}
\usepackage{multicol}
\usepackage{longtable,caption}
\usepackage{multicap}
\usepackage{tabu}
\usepackage{amsmath}
\usepackage[left = 15mm, right = 15mm, top = 25mm, bottom= 20mm]{geometry}
\usepackage{stmaryrd, amssymb}
\usepackage{comment}
\usepackage{changepage}

\usepackage{pgf}
\usepackage{tikz}
\usetikzlibrary{positioning} % Necessary to use right=5cm of A
\usetikzlibrary{arrows,automata}
\newcommand{\ksec}[2]{\noindent \textbf{\large #1} #2\par}
\newcommand{\ksolve}[1]{\noindent\textbf{\large Solution #1: }\par}
\newcommand{\arithminus}{\dot{\ \rule[2.5pt]{6pt}{0.5pt}\ }}
\newcommand{\smallarithminus}{\dot{\rule[1pt]{3pt}{0.5pt}}}
\newcommand{\absminus}{\dot{\ \rule[2.5pt]{3.8pt}{0.6pt}\ }\!\!\!\!\!\dot{\ \rule[2.5pt]{3.8pt}{0.6pt}\ }}
\newcommand{\kspace}{\vspace{0.5cm}}

\begin{document}
	
	\section*{稠密无端点全序}
	
	\ksec{定义}{\texttt{DLWP}理论包括如下语句:}
	\begin{itemize}
		\item $ \forall x:\ \lnot (x \textless x) $
		\item $ \forall x\ y:\ x \textless y \lor y \textless x \lor x = y$	
		\item $ \forall x\ y\ z:\ x < y \land y < z \implies x < z $
		\item $ \forall x \ y: \exists z:\ x < y \implies x < z < y $
		\item $ \forall x\ \exists y\ z:\ y < x < z $
	\end{itemize}
	注:前三条件代表全序
	
	\ksec{\texttt{DLWP}模型举例}{}
		是$ (\mathbb{Q}, \textless), (\mathbb{R}, \textless), ((0, 1), \textless) $
		
		非$ (\mathbb{N}, \textless), ({(0, 1]}, \textless) $
		
	
	\ksec{定理:所有满足\texttt{DLWP}且可数的模型都是同构的}{}
	证明(back and forth):
	不妨设两个满足题设的模型$ \mathcal{M}\ \mathcal{N} $, 由于二者可数,其定义域枚举如下:
		$$
		\begin{array}{l}
			M: m_1, ...., m_k, ...\\
			N: n_1, ...., n_l, ...\\
		\end{array}
		$$
		
	定义$ f: M \rightarrow N$
		\begin{itemize}
			\item 第一步:$ f(m_1) = n_1 $
			\item k+1步:
				\begin{itemize}
					\item 若$ m_{k+1} \in \texttt{dom}(f) $, 跳过此小步; 否则:将已定义原像按序枚举为$  m_{i1} \textless m_{i2} \textless ...\textless m_{ix}  $,对应映射值 $ f(m_{i1}), f(m_{i2}), ..., f(m_{ix}) $,因我们之前定义的方式是保序的,故$ f(m_{i1}) \textless f(m_{i2}) \textless ... \textless f(m_{ix}) $;若:
					\begin{itemize}
						\item $ m_{k+1} > m_{ix}$: 由$ \mathcal{N} \models \texttt{DLWP} $,故存在$ z \in N $使得$ f(m_{ix}) < z $成立,定义$ f(m_i) = z $
						\item $ m_{k+1} < m_{i1} $: 同理存在$ y \in N $使得$ f(m_{i1}) > y $成立,定义$ f(m_i) = y $
						\item $ m_{i(j)} < m_{k+1} < m_{i(j+1)} $, 则存在$ z \in N $使得$ f(m_{i(j)}) < z < f(m_{i(j+1)}) $, 定义$ f(m_i) = z $
					\end{itemize}
				
					\item 同理再看$ n_{k+1} $, 不属于$ \texttt{range}(f) $跳至下一步,否则:将已定义像按序枚举为$ n_{i1}, n_{i2}, ..., n_{ix} $, 对应逆映射值$ f^{-1}(n_{i1}), f^{-1}(n_{i2}), ..., f^{-1}(n_{iy}) $,有$ f^{-1}(n_{i1}) < f^{-1}(n_{i2}) < ... < f^{-1}(n_{iy}) $. 若:
					\begin{itemize}
						\item $ n_{k+1} > n_{iy}$: 由$ \mathcal{M} \models$\texttt{DLWP},故存在$ z \in M $使得$ f^{-1}(n_{ix}) < z $成立,定义$ f(z) = n_{k+1} $
						\item $ n_{k+1} < n_{i1} $: 同理存在$ y \in M $使得$ f(n_{i1}) > z $成立,定义$ f(z) = n_{k+1}  $
						\item $ n_{i(j)} < n_{k+1} < n_{i(j+1)} $, 则存在$ z \in M $使得$ f^{-1}(n_{i(j)}) < z < f^{-1}(n_{i(j+1)}) $, 定义$ f(z) = n_{k+1} $
					\end{itemize}
				\end{itemize}
		\end{itemize}
	
	由定义知$ f $是保序双射,故$ \mathcal{M} \cong \mathcal{N} $
	
\end{document}