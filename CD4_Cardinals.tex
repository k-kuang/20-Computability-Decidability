\documentclass[UTF8, 9pt, a4paper]{ctexart}
\usepackage[utf8]{inputenc}
\usepackage{multicol}
\usepackage{longtable,caption}
\usepackage{multicap}
\usepackage{tabu}
\usepackage{amsmath}
\usepackage[left = 15mm, right = 15mm, top = 25mm, bottom= 20mm]{geometry}
\usepackage{stmaryrd, amssymb}
\usepackage{comment}
\usepackage{changepage}

\usepackage{pgf}
\usepackage{tikz}
\usetikzlibrary{positioning} % Necessary to use right=5cm of A
\usetikzlibrary{arrows,automata}
\newcommand{\ksec}[2]{\noindent \textbf{\large #1} #2\par}
\newcommand{\ksolve}[1]{\noindent\textbf{\large Solution #1: }\par}
\newcommand{\arithminus}{\dot{\ \rule[2.5pt]{6pt}{0.5pt}\ }}
\newcommand{\smallarithminus}{\dot{\rule[1pt]{3pt}{0.5pt}}}
\newcommand{\absminus}{\dot{\ \rule[2.5pt]{3.8pt}{0.6pt}\ }\!\!\!\!\!\dot{\ \rule[2.5pt]{3.8pt}{0.6pt}\ }}
\newcommand{\kspace}{\vspace{0.5cm}}

\begin{document}
	
	$ \mathcal{O} $是序数集合(Ordinals),$ \mathcal{C} $代表基数集合(Cardinals)
	
	\section{基数}
	
	\ksec{定义:}{}
		$  \alpha \in \mathcal{C} \texttt{ if } \alpha \in \mathcal{O} \land \forall \beta \in \mathcal{O}:\ \beta < \alpha \implies \overset{=}{\beta} < \overset{=}{\alpha} $
	
	\ksec{基数与序数的讨论}{}
	\begin{itemize}
		\item 有穷上:不同序数必然对应不同序数\\
			例:1对1,n对n; 故有穷上每个序数都是基数
		\item 无穷上:不同序数可能对应同一基数\\
			例:$ \omega, \omega+1 $都对$ \aleph_0 $(易证$ \omega,\omega+1 $等势(cardinality)); 最小的无穷基数是$ \aleph_0 $
	\end{itemize}
	
	\ksec{我对$ \aleph_0, \aleph_1... $的理解}{}
	$ \aleph_0 = \omega = \{\omega, \omega + 1, \omega + 2, ... ,\omega + n\} $中最小的一个
	
	$ \aleph_1 = \omega + \omega = \{ \omega + \omega, \omega + \omega + 1, ..., \omega + \omega + n \}$中最小的一个
	
	...
	
	\ksec{$\aleph_1$为何存在?}{5:15}
	定义$ A=\{ \alpha \mid \alpha >\aleph_0 \} $, 存在一个$ \alpha_0 \in A $, 由集合定义,对任意$ \beta < \alpha_0$, $ \beta \leq \aleph_0 \textless \alpha_0 $成立。$ \alpha_0 $即为$ \aleph_1 $. 对一般的$ \kappa $也有$ \kappa^{+} $(即$ \kappa $后面的一个基数)存在
	
	\ksec{极限基数与后继基数}{}
	\begin{itemize}
		\item Limit Cardinal: $ \forall  $\\
		例:$ \omega $, 由$ \forall n < \omega: \ n^{+} < \omega $; 又例$ \aleph_\omega, \aleph_{\omega + \omega},... $
		\item Successor Cardinal: $ \exists \kappa:\ \kappa = \lambda^{+} $\\
		例:$ \aleph_1, \aleph_2, ... $
	\end{itemize}


	\ksec{并非简单自然数推广:共尾性}{}
	$$ \texttt{cf}(\kappa) \triangleq \texttt{min}\{ \alpha \mid \exists f: \alpha \rightarrow \kappa :\ \forall \lambda < \kappa:\ \exists \beta < \alpha: f(\beta) > \lambda \} $$
	
	注解:$ f $理解为$ \alpha $到$ \kappa $上的无界函数。
	
	
	\textbf{试求\texttt{cf}$ (\aleph_\omega) $}
	\begin{itemize}
		\item 定义$ f: \omega \rightarrow \aleph_\omega $为:
		$$ f(n) = \aleph_n $$
		
		由$ f $无界,故$ \texttt{cf}(\aleph_\omega) \leq \omega $
		
		\item 假设$ \texttt{cf}(\aleph_\omega) < \omega $, 则存在$ n $使得$ n $到$ \aleph_\omega $的无界映射$ f $存在,不失一般性定义$ f $为:
		
		$$ f(i) = \aleph_{g(i)}  \qquad \texttt{其中}g\texttt{为有界函数}$$
		
		则$ \forall i < n:\ f(i) < \aleph_{\sum\limits_{j < n}g(j) + 1} < \aleph_\omega $, 与$ f $是无界函数矛盾。
	\end{itemize}

	因此$ \texttt{cf}(\aleph_\omega) = \omega $
	
	\vspace{0.3cm}
	
	满足$ \texttt{cf}(\kappa) = \kappa $的$ \kappa $称正则基数。$ \texttt{cf}(\aleph_0) = \omega $故$ \aleph_0 $是正则基数。$ \aleph_\omega $不满足此性质, 称奇异基数。
	
	
	\ksec{良序与良序公理(well-ordering)}{}
	良序:$$ \begin{array}{ll}
		\texttt{well-order}(\langle A, \preceq \rangle) \triangleq \texttt{total}(\langle A, \prec \rangle) \land (\forall B \subseteq A:\ B \neq \emptyset \implies \exists b:\ \texttt{min}_\preceq(B, b))\\
		
		\texttt{total}(\langle A, \preceq \rangle) \triangleq (\forall x\ y \in A:\ x \preceq y \lor y \preceq x) \\
		\qquad \qquad \qquad \qquad \land (\forall x\ y \in A:\ x \preceq y \land y \preceq x \implies x = y) \\
		\qquad \qquad \qquad \qquad \qquad \land (\forall x\ y\ z \in A:\ x \preceq y \land y \preceq z \implies x \preceq z)\\
		
		\texttt{min}_\preceq(B, b) \triangleq \forall x \in B:\ b \preceq x
	\end{array} $$
	
	
	良序公理:$$ \forall A:\ \exists \alpha \in \mathcal{O}:\ \exists f: \alpha \rightarrow A\texttt{是双射} $$
	
	注解:序数及其关系是全序且良序,此公理即意为对任何一个集合A,都可以定义A上的一个全且良的关系。
	
	\ksec{集合的基数}{}
	
	由良序公理能定义任何一个集合$ A $的基数(之前定义的是序数集合上的)
	$$ |A| = \texttt{min}\{\alpha \mid \overset{=}{\alpha} = \overset{=}{A}\} $$
	
	证明$ |A| $是基数:1. 根据良序公理$ |A| $存在. 2. 假设$ |A| $不是基数,则存在$ \beta < |A| $使得$ \overset{=}{\beta} = \overset{=}{A} $成立,故$ |A| \leq \beta $矛盾。
	
	例:$ |\mathbb{N}| = |\mathbb{Z}| = |\mathbb{Q}| = \aleph_0$
	
	
	\ksec{漂亮的闭集刻画:完备集定理}{非常重要!!!}
		$$ \texttt{任何一个不可数闭集都是一个可数集和一个完备集的不交并} $$
	
	证明:

	\ksec{利用良序公理计算集合的基数}{}
	此例题用于阐述如何计算基数:证明任意无穷集合$ A $有 $ |A \times A| = |A| $成立
	
	证明:
	
	
	
	\pagebreak	
	
	\ksec{练习:}{}
	试求\texttt{cf}($ \aleph_1 $). 
	证明$ \texttt{cf}(\aleph_n) = \aleph_n $, 所有后继基数都是正则基数。
	
	
\end{document}